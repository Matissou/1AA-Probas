\documentclass{article}
\usepackage[utf8]{inputenc}
\usepackage[T1]{fontenc}
\usepackage[french]{babel}
\usepackage{minted}
\usepackage{amssymb}
\usepackage{amsmath}
\usepackage{geometry}
\usepackage{textcomp}
\newcommand{\textapprox}{\raisebox{0.5ex}{\texttildelow}}

  % added for comments and todo
\usepackage[colorinlistoftodos]{todonotes}


\geometry{total={210mm,297mm},
left=25mm,right=25mm,
top=25mm,bottom=25mm}

\title{Probabilités}
\date{14 mai 2017}

\begin{document}
\maketitle

\section{Eemple de lois continues}

\subsection{Loi uniforme sur (a,b)}

$a < b$


$a, b \in \mathbb{R}$

\begin{itemize}
  \item Densité : $f(x) = \frac{1}{b-a} \mathbb{1}_{a,b}(x)$
\item Fonction de répartition :
\begin{align*}
  F(x) &= \int_{- \infty}^{x} f(t)dt \\
  &= 0 \textit{ si } x \leq a \\
  &= \frac{x-a}{b-a} \textit{ si } a < x \leq b \\
  &= 1 \textit{ si } x > b
\end{align*}
\item
\begin{align*}
  \mathbb{E}(X) &= \int_{- \infty}^{+ \infty}xf(x)dx = \int_{a}^{b} \frac{x}{b-a}dx \\
  &= [ \frac{x}{2(b-a)}]_{a}^b \\
  &= \frac{b^2 - a^2}{2(b-a)} \\
  &= \frac{a+b}{2}
\end{align*}
\item
\begin{align*}
  \mathbb{E}(X^2) &= \int_{- \infty}^{+ \infty}x^2f(x)dx \\
  &= \int_{a}^{b} \frac{x^2}{b-a} dx\\
  &= [ \frac{x^3}{3(b-a)}]_{a}^b \\
  &= \frac{b^3-a^3}{3(b-a)} \\
  &= \frac{b^2+ab+a^2}{3}
\end{align*}
\item
\begin{align*}
  \textit{Var}(X) &= \frac{1}{12}(b^2-2ab+a^2) \\
  &= \frac{1}{12}(b-a)^2
\end{align*}
\end{itemize}

\subsection{Passage d'une uniforme centrée réduite à une uniforme (a,b)}
Si $U \textapprox \textit{Unif}(0,1)$, on pose :

$X = (b-a)U + a \textapprox \textit{Unif}(a,b)$

\section{Loi exponentielle}

$X \textapprox \textit{Exp}(\lambda) \textit{ et } \lambda > 0$

\subsection{Densité}
$$ f_X(x) = \lambda e^{-\lambda x} \textit{ pour } x > 0 \textit{ sinon } 0$$

\subsection{FDR}

$$ F_X(x) = 1 - e^{-\lambda x} $$ si $x > 0$, $0$ sinon.

\subsection{E et V}

$\mathbb{E}(X) = \frac{1}{\lambda}$

$\textit{Var}(X) = \frac{1}{\lambda^2}$

\subsection{Propriétés}

Si $X \textapprox \textit{Exp}(1)$, quelle est la loi de $\frac{1}{\lambda}X$ ?

$$ F_Y(x) = \mathbb{P}(Y \leq x) = \mathbb{P}(\frac{1}{\lambda} X \leq x) = \mathbb{P}(X \leq x) $$
$$ = F_X(\lambda x)$$

Donc $Y \textapprox \textit{Exp}(\lambda)$

Soit $U \textapprox \textit{Unif}(0,1)$,
$ Y = - \frac{1}{\lambda} ln(U)$

$$ F_Y(x) = \mathbb{P}(- \frac{1}{\lambda}ln(U) \leq X)$$
$$ = 0 \textit{ si } x \leq 0 $$

\end{document}